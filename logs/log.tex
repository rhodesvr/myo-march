\documentclass{article}
\usepackage{amssymb}
\usepackage{amsmath}
\usepackage{amsfonts}
\usepackage[margin = 1in]{geometry}
\newtheorem{theorem}{Theorem}

\begin{document}

\title{
  Fall 2015 CS Log
}

\author{
Preston Tunnell Wilson }

\date{}

\maketitle

\section*{November 21st}
Trying out two MYO's!
\section*{November 5th}
It has been a pretty active day today!
I was looking over the MYO code and how we could change it to our three pronged approach to experiements,
but I realized that there might be a discrepancy between the version on one computer and the version on the other.
Thus, I planned to checkout the various commits and see which one worked.

As it worked out, the most recent version, or the most recent commit, was incorrect.
After double checking several times, I assured myself of this.
Now that we know this, we can figure out how this code works better
which will lead to a cleaner and working three-pronged experiment.
To be sure that I don't accidentally overwrite/lose this working version,
I have put it in its own branch.

Additionally, we met with Bryton today.
He showed us his code-there are three files in Documents>JogJoy (or something like that).
\begin{itemize}
\item ExpLeanJoy uses head proximity to the Kinect after taking a baseline of the subject's posture.
  Leaning one's head side to side rotates the view.
\item SitJoy is meant to be navigation without leaning.
  Instead, it uses hand/wrist movement.
  Additionally, it has a speed count to ramp up movement.
\item Finally, ExpJogJoy is the standing up version.
  However, much like my code, it ran into problems of occlusion in the Kinects.
  Furthermore, people would sometimes be facing the wrong direction after turning around,
  thus switching the left and right arms.
\end{itemize}

\section*{October 31st}
Ansel and I met with Will and Josh!
They showed us that Myo2 and Myo should be the codes which we should run.
Additionally, they pointed out that we have to have an oculus attached to the computer to get it to work.
We might also need to end the Myo task in order to get it working.
Additionally, at the end of the experiment, we have to have the hub shut down.
They think it is a keypress of ``p''; we could additionally stop the script.
If we are having trouble getting the Myo to connect, we can create a new profile instead.
That might be faster.
\section*{October 11th}
Ansel and I stopped by the VE lab today.
Unfortunately, we did not get much accomplished.
After fumbling through getting the myo arm bands working,
we still ran into problems getting the experiments up and running.
We contacted Bryton to see if he would be able to help us with getting
the experiments working.

I plan on restructuring our experiment code into three classes.
We delegate knowledge of the experiment, knowledge of navigation, and knowledge of the environment
into three separate classes.
One issue is which class should take responsibility for moving through the different trials.
The environment knows where the cylinders and objects are...
I guess it will let the experiment know when it is done with a specific location and then with a trial.
\end {document}
