\documentclass{article}
\usepackage{amssymb}
\usepackage{amsfonts}
\usepackage{amsmath,amsthm}
\usepackage{enumitem}
\usepackage[margin = 1in]{geometry}
\newtheorem{theorem}{Theorem}
\theoremstyle{definition}
\newtheorem*{prob}{Problem}
\newtheorem*{answer}{Answer}
\newtheorem*{ex}{Exercise}
\newtheorem*{prop}{Proposition}

\def\neg{^{-1}}
\def\Z{\mathbb Z}
\def\N{\mathbb N}
\def\Q{\mathbb Q}
\def\R{\mathbb R}
\def\e{\epsilon}
\def\RxR{\mathbb R \times \mathbb R}
\def\F{\mathcal F}

\ifx \ser \undefined
\newcommand \ser[2][i]{\{#2_{#1}\}_{#1=1}^\infty}
% use like
% $\ser{W}$
% $\ser[j]{N}$
\fi

\begin{document}

\title{
  Comparing Various Locomotion Methods within Virtual Environments
}

\author{
  Preston Tunnell Wilson
}

\date{
  \today
}

% check out website at https://fs22.formsite.com/webmanagerrhodesedu/form7/index.html

\maketitle

\begin{itemize}
\item Faculty Sponsor: Betsy Sanders
\item Other Authors: Ansel MacLaughlin, Will Kalescky
\item Title: See above
\item Division: Science
\item Presentation preference (oral | poster): oral
\item Abstract (max 200 words):

  Two inexpensive methods of exploring a virtual environment are ``walking in place'' (WIP) and arm swinging.
  These techniques are compelling because
  they seem to provide more proprioceptive cues than traditional
  inexpensive virtual navigation techniques such as joysticks or controllers.
  Specifically, WIP and arm swinging seem to result in better spatial awareness of the environment.
  For example, in our prior work, we had success in implementing a WIP method
  using an inexpensive Nintendo Wii Balance Board
  and later with Microsoft Kinect sensors.
  We showed that participants' spatial orientation was the same as normal walking
  and superior to joystick navigation.
  We used an inexpensive wearable device called the Myo armband (199 USD)
  to implement a simple arm swinging algorithm that
  allows a user to freely explore an HMD-based virtual environment.
  We found that our arm swinging method outperforms a simple joystick
  and that spatial orientation is comparable to physically walking on foot.
  We expand on this work by comparing many locomotion methods to each other:
  WIP to arm swinging to physical locomotion;
  seated joystick to standing joystick;
  and other permutations as we obtain our results.

\item The work discussed in the abstract was performed as part of a Rhodes Fellowship:
  Yes
\end{itemize}

\end {document}

